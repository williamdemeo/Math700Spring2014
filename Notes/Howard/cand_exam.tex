%%
%% This is LaTeX2e input
%%

\documentclass[11pt]{amsart}
\pagestyle{plain}


\author{}
\date{}
\title{Linear Algebra Questions from the Admission to Candidacy Exam} 
%\textheight=9in:Swedish paper
%\textheight=8.2in%American paper
%\textwidth=6.1in
%\hoffset=-.25in
%\voffset=-1in

%\input ts:intdef

\renewcommand{\(}{\left(}
\renewcommand{\)}{\right)}
\renewcommand{\[}{\left[}
\renewcommand{\]}{\right]}


%\newcommand{\beq}{\begin{equation}}
%\newcommand{\eeq}{\end{equation}}


\newfont{\bi}{cmbxti10 scaled \magstep1}
\newcommand{\cl}{\setlength{\itemsep}{-5pt}
                 \setlength{\parsep}{-5pt}}


\setlength{\partopsep}{3pt}
\setlength{\topsep}{3pt}


\renewcommand{\emptyset}{\varnothing}
\renewcommand{\phi}{\varphi}


\newcommand{\area}{\mathop{\rm Area}}
\newcommand{\ba}{\begin{array}}
\newcommand{\cn}{\mbox{\bf C}}
\newcommand{\cd}{,\ldots,}
\newcommand{\ch}{{\cal H}}
\newcommand{\compsub}{\mbox{$\subset\subset$}}
\newcommand{\comsub}{\mbox{$\subset\subset$}}
\newcommand{\dist}{\mathop{\rm dist}}
\newcommand{\D}{\Delta}
\newcommand{\e}{\varepsilon}
\newcommand{\E}{\mbox{\bf E}}
\newcommand{\ea}{\end{array}}
\newcommand{\f}{\partial}
\newcommand{\g}{\gamma}
\newcommand{\ga}{\alpha}
\newcommand{\gb}{\beta}
\newcommand{\gd}{\delta}
\newcommand{\gi}{\iota}
\newcommand{\grad}{\nabla}
\newcommand{\grg}{\mbox{\bf g}}
\newcommand{\hil}{{\cal H}}
\newcommand{\id}{\mathop{\rm Id}\nolimits}
\newcommand{\im}{\mathop{\rm Im}}
\newcommand{\image}{\mathop{\rm Image}\nolimits}
\newcommand{\kernel}{\mathop{\rm Kernel}\nolimits}
\newcommand{\la}{\langle}
\newcommand{\lap}{\Delta}
\newcommand{\length}{\mathop{\rm Length}\nolimits}
\newcommand{\m}{\hskip-8.5pt&{}&\hskip-8.5pt}
\newcommand{\nn}{\nonumber}
\newcommand{\oc}{^\bot}
\newcommand{\om}{\Omega}
\newcommand{\oo}{\mbox{\bf o}}
\newcommand{\opnorm}[1]{\|#1\|_{\mathop{\rm Op}}}
\newcommand{\R}{{\mathbf R}}
\newcommand{\ra}{\rangle}
\newcommand{\s}{\sigma}
\newcommand{\sE}{{\small \bf E}}
\newcommand{\sr}{{\small \bf R}}
\newcommand{\rank}{\mathop{\rm rank}}
\newcommand{\tr}{\mathop{\rm Tr}}
\newcommand{\trace}{\mathop{\rm trace}}
\newcommand{\uu}{\omega}
\newcommand{\vol}{\mathop{\rm Vol}\nolimits}
\newcommand{\w}{\wedge}
\newcommand{\W}{\bigwedge}	
\newcommand{\ww}{\omega}
\newcommand{\x}{\xi}
\newcommand{\y}{\eta}
\newcommand{\z}{\mbox{\bf Z}}


\newtheorem{thm}{Theorem}[section]
\newtheorem{lemma}[thm]{Lemma}
\newtheorem{prop}[thm]{Proposition}
\newtheorem{cor}[thm]{Corollary}
\newtheorem{remark}[thm]{Remark}
\newtheorem{definition}[thm]{Definition}
\newtheorem{notation}[thm]{Notation}
\newtheorem{exercise}[thm]{Exercise}
\newtheorem{prob}{Problem}
\newtheorem{example}[thm]{Example}
\newtheorem{step}{Step}

\setlength{\textheight}{9in}%Amercian paper
%\setlength{\textheight}{9.9in}%Swedish paper
\setlength{\textwidth}{6.7in}
\setlength{\hoffset}{-.5in}
\setlength{\voffset}{-1in}



%\addtolength{\itemsep}{-10pt}


%\newlength{\myitemsep}{1in}

%\renewcommand{\itemsep}{\myitemsep}

%\setlength{\itemsep}{-10pt}


%\input intdef
%\newcommand{\mod}{\mathop{\rm mod}\nolimits}

\newcommand{\heading}[1]{\centerline{\large\sc #1}}

\newcommand{\num}{\begin{enumerate}}
\newcommand{\enum}{\end{enumerate}}
\newcommand{\fe}{{\bf F}}
\renewcommand{\tr}{\mathop{\rm tr}\nolimits}
\hoffset=-1in


\begin{document}
\maketitle

\noindent
The following is a more or less complete list of the linear algebra
questions that have appeared on the admission to candidacy exam for
the last ten years.  Some questions have been reworded a little.
\bigskip

\heading{January 1984}
\num
\item Let $V$ be a finite-dimensional vector space and let $T$ be a
linear operator on $V$.  Suppose that $T$ commutes with every
diagonalizable linear operator on $V$.  Prove that $T$ is a scalar
multiple of the identity operator.
\item Let $V$ and $W$ be vector spaces and let $T$ be a linear
operator from $V$ into $W$.  Suppose that $V$ is finite-dimensional.
Prove $\mathop{\rm rank}(T)+\mathop{\rm nullity}(T)=\dim V$.
\item Let $A$ and $B$ be $n\times n$ matrices over a field ${\bf F}$.
\num
\item Prove that if $A$ or $B$ is nonsingular, then $AB$ is similar to
$BA$.
\item Show that there exist matrices $A$ and $B$ so that $AB$ is {\em
not} similar to $BA$.
\item What can you deduce about the eigenvalues of $AB$ and $BA$?
Prove your answer.
\enum
\item Let $A=\(\begin{array}{cc}D&E\\F&G\end{array}\)$, where $D$ and
$G$ are $n\times n$ matrices.  If $DF=FD$ prove that $\det
A=\det(DG-FE)$.
\item If ${\bf F}$ is a field, prove that every ideal in ${\bf F}[x]$
is principal.
\enum
\heading{August 1984}

\num 
\item Let $V$ be a finite dimensional vector space.  Can $V$ have
three distinct proper subspaces $W_0$, $W_1$ and $W_2$ such that
$W_0\subseteq W_1$, $W_0+W_2=V$, and $W_1\cap W_2=\{0\}$?
\item Let $n$ be a positive integer.  Define
\begin{eqnarray*}
G=\m \{A: \mbox{ $A$ is an $n\times n$ matrix with only integer
entries and $\det A\in\{-1,+1\}$\}},\\
H=\m \{A: \mbox{ $A$ is an invertible $n\times n$ matrix and both $A$
and $A^{-1}$ have only integer entries}\}.
\end{eqnarray*} 
Prove $G=H$.
\item Let $V$ be the vector space over $\R$ of all $n\times n$
matrices with entries from $\R$.
\num
\item Prove that $\{I,A,A^2\cd A^n\}$ is linearly dependent for all
$A\in V$.
\item Let $A\in V$.  Prove that $A$ is invertible if and only if $I$
belongs to the span of $\{A, A^2\cd A^n\}$.
\enum
\item Is every $n\times n$ matrix over the field of complex numbers
similar to a matrix of the form $D+N$, where $D$ is a diagonal matrix,
$N^{n-1}=0$, and $DN=ND$?  Why?
\enum

\heading{January 1985}
\num \item \num
\item Let $V$ and $W$ be vector spaces and let $T$ be a linear
operator from $V$ into $W$.  Suppose that $V$ is finite-dimensional.
Prove $\mathop{\rm rank}(T)+\mathop{\rm nullity}(T)=\dim V$.
\item  Let $T\in L(V,V)$, where $V$ is a finite dimensional vector space.
(For a linear operator $S$ denote by ${\mathcal N}(S)$ the null space and
by ${\mathcal R}(S)$ the range of $S$.)
\num
\item Prove there is a least natural number $k$ such that
${\mathcal N}(T^{k})= {\mathcal N}(T^{k+1})={\mathcal N}(T^{k+2})\cdots$  Use
this $k$ in the rest of this problem.
\item Prove that ${\mathcal R}(T^k)={\mathcal R}(T^{k+1})={\mathcal
R}(T^{k+2})\cdots$
\item Prove that ${\mathcal N}(T^k)\cap {\mathcal R}(T^k)=\{0\}$.
\item Prove that for each $\alpha\in V$ there is exactly one vector in
$\alpha_1\in{\mathcal N}(T^k)$ and exactly one vector $\alpha_2\in{\mathcal
R}(T^k)$ such that $\alpha=\alpha_1+\alpha_2$.
\enum
\enum
\item Let $\fe $ be a field of characteristic~0 and let 
$$
W=\left\{A=[a_{ij}]\in \fe^{n\times n}: \tr(A)=\sum_{i=1}^na_{ii}=0\right\}.
$$
For $i,j=1\cd n$ with $i\ne j$, let $E_{ij}$ be the $n\times n$ matrix
with $(i,j)$-th entry~1 and all the remaining entries~0.  For $i=2\cd
n$ let $E_{i}$ be the $n\times n$ matrix with $(1,1)$~entry $-1$,
$(i,i)$-th entry~$+1$, and all remaining entries~0.  Let
$$
S=\left\{E_{ij}: i,j=1\cd n\ \mbox{and}\ i\ne j\right\}\cup
\left\{E_{i}: i=2\cd n\right\}.
$$
[{\sc Note:} You can assume, without proof, that $S$ is a linearly
independent subset of $\fe^{n\times n}$.]
\num
\item Prove that $W$ is a subspace of $\fe^{n\times n}$ and that
$W=\mathop{\rm span}(S)$.  What is the dimension of $W$?
\item Suppose that $f$ is a linear functional on $\fe^{n\times n}$ such
that 
\num
\item $f(AB)=f(BA)$, for all $A,B\in \fe^{n\times n}$.
\item $f(I)=n$, where $I$ is the identity matrix in $\fe^{n\times n}$.
\enum
\enum
Prove that $f(A)=\tr(A)$ for all $A\in\fe^{n\times n}$.
\enum
\heading{August 1985}
\num
\item Let $V$ be a vector space over $\cn$.  Suppose that $f$ and $g$
are linear functionals on $V$ such that the functional
$$
h(\alpha)=f(\alpha)g(\alpha)\quad \mbox{for all}\quad \alpha\in V
$$
is linear.  Show that either $f=0$ or $g=0$.
\item Let $C$ be a $2\times2$ matrix over a field $\fe$.  Prove: There
exists matrices $C=AB-BA$ if and only if $\tr(C)=0$.
\item Prove that if $A$ and $B$ are $n\times n$ matrices from $\cn$ and
$AB=BA$, then $A$ and $B$ have a common eigenvector.
\enum

\heading{January 1986}
\num
\item Let $\fe$ be a field and let $V$ be a finite dimensional vector
space over $\fe$.  Let $T\in L(V,V)$.  If $c$ is an eigenvalue of $T$,
then prove there is a nonzero linear functional $f$ in $L(V,\fe)$ such
that $T^*f=cf$. (Recall that $T^*f=fT$ by definition.)
\item Let $\fe$ be a field, $n\ge 2$ be an integer, and let $V$ be the
vector space of $n\times n$ matrices over \fe.  Let $A$ be a fixed
element of $V$ and let $T\in L(V,V)$ be defined by $T(B)=AB$.
\num
\item Prove that $T$ and $A$ have the same minimal polynomial.
\item If $A$ is diagonalizable, prove, or disprove by counterexample,
that $T$ is diagonalizable.
\item Do $A$ and $T$ have the same characteristic polynomial?  Why or
why not?
\enum
\item Let $M$ and $N$ be $6\times 6$ matrices over $\cn$, both having
minimal polynomial $x^3$.
\num
\item Prove that $M$ and $N$ are similar if and only if they have the
same rank.
\item Give a counterexample to show that the statement is false if 6
is replaced by 7.
\enum
\enum
\heading{ August 1986} 
\num
\item Give an example of two $4\times 4$ matrices that are not similar
but that have the same minimal polynomial.
\item Let $(a_1,a_2\cd a_n)$ be a nonzero vector in the real
$n$-dimensional space~$\R^n$ and let $P$ be the hyperplane
$$
\left\{(x_1,x_2\cd x_n)\in \R^n: \sum_{i=1}^na_ix_i=0\right\}.
$$
Find the matrix that gives the reflection across $P$.
\enum

\heading{January 1987}
\num
\item
Let $V$ and $W$ be finite-dimensional vector spaces and let $T:V\to
W$ be a linear transformation.  Prove that that exists a basis $\mathcal
A$ of $V$ and a basis $\mathcal B$ of $W$ so that the matrix $[T]_{\mathcal A,
B}$ has the block form $\[\begin{array}{cc}I&0\\0&0\end{array}\]$.
\item Let $V$ be a finite-dimensional vector space and let $T$ be a
diagonalizable linear operator on~$V$.  Prove that if $W$ is a
$T$-invariant subspace then the restriction of $T$ to $W$ is also
diagonalizable. 
\item Let $T$ be a linear operator on a finite-dimensional vector.
Show that if $T$ has no cyclic vector then, then there exists an
operator $U$ on $V$ that commutes with $T$ but is {\em not} a
polynomial in $T$.
\enum

\heading{August 1987}
\num
\item Exhibit two real matrices with no real eigenvalues which have
the same characteristic polynomial and the same minimal polynomial but
are not similar.
\item Let $V$ be a vector space, not necessarily finite-dimensional.
Can $V$ have three distinct proper subspaces $A$, $B$, and $C$, such
that $A\subset B$, $A+C=V$, and $B\cap C=\{0\}$?
\item Compute the minimal and characteristic polynomials of the
following matrix.  Is it diagonalizable?
$$
\[\begin{array}{cccc}5&-2&0&0\\ 6&-2&0&0\\0&0&0&6\\0&0&1&-1\end{array}\]
$$
\enum
\heading{August 1988}
\num\item\num
	\item Prove that if $A$ and $B$ are linear transformations on
an $n$-dimensional vector space with $AB=0$, then $r(A)+r(B)\le n$
where $r(\cdot)$ denotes rank.
	\item For each linear transformation $A$ on an $n$-dimensional
vector space, prove that there exists a linear transformation $B$ such
that $AB=0$ and $r(A)+r(B)=n$.
\enum
\item \num\item Prove that if $A$ is a linear transformation such that
$A^2(I-A)=A(I-A)^2=0$, then $A$ is a projection.
\item Find a non-zero linear transformation so that $A^2(I-A)=0$ but
$A$ is {\em not\/} a projection.
\enum
\item If $S$ is an $m$-dimensional vector space of an $n$-dimensional
vector space $V$, prove that $S^\circ$, the annilihilator  of $S$, is
an $(n-m)$-dimensional subspace of $V^*$.
\item Let $A$ be the $4\times 4$ real matrix
$$
A=\[\begin{array}{cccc} 1&1&0&0\\-1&-1&0&0\\-2&-2&2&1\\ 1&1&-1&0\end{array}\]
$$
\num
\item Determine the rational canonical form of $A$.
\item Determine the Jordan canonical form of $A$.
\enum
\enum

\heading{January 1989}
\num \item
Let $T$ be the linear operator on $\R^3$ which is represented by
$$
A=\[\begin{array}{ccc}1&1&-1\\1&1&-1\\1&0&0\end{array}\]
$$
in the standard basis.  Find matrices $B$ and $C$ which represent
respectively, in the standard basis, a diagonalizable linear operator $D$
and a nilpotent linear operator $N$ such that $T=D+N$ and $DN=ND$.
\item Suppose $T$ is a linear operator on $\R^5$ represented in some
basis by a diagonal matrix with entries $-1$, $-1$, 5, 5, 5 on the
main diagonal.
\num
\item Explain why $T$ can not have a cyclic vector.
\item Find $k$ and the invariant factors $p_i=p_{\alpha_i}$ in the
cyclic decomposition $\R^5=\bigoplus_{i=1}^k Z(\alpha_i;T)$.
\item Write the rational canonical form for $T$.
\enum
\item Suppose that $V$ in an $n$-dimensional vector space and $T$ is a
linear map on $V$ of rank~1.  Prove that $T$ is nilpotent or
diagonalizable. 
\enum

\heading{August 1989}

\num
\item Let $M$ denote an $m\times n$ matrix with entries in a field.
Prove that
\begin{eqnarray*}
\m\mbox{the maximum number of linearly independent rows of $M$}\\
=\m\mbox{the maximum number of linearly independent columns of $M$}
\end{eqnarray*}
(Do not assume that $\mathop{\rm rank} M=\mathop{\rm rank}M^t$.)
\item Prove the Cayley-Hamilton Theorem, using only basic properties
of determinants.
\item Let $V$ be a finite-dimensional vector space.  Prove there a linear
operator $T$ on $V$ is invertible if and only if the constant term in
the minimal polynomial for $T$ is non-zero.
\item\num\item Let
$M=\[\begin{array}{ccc}0&1&0\\0&0&1\\1&1&-1\end{array}\]$. Find a
matrix $T$ (with entries in $\cn$) such that $T^{-1}MT$ is diagonal, or
prove that such a matrix does not exist.
\item Find a matrix whose minimal polynomial is $x^2(x-1)^2$, whose
characteristic polynomial is $x^4(x-1)^3$ and whose rank is $4$.
\enum
\item Suppose $A$ and $B$ are linear operators on the same
finite-dimensional vector space $V$.  Prove that $AB$ and $BA$ have
the same characteristic values.
\item Let $M$ denote an $n\times n$ matrix with entries in a field
$\fe$.  Prove that there is an $n\times n$ matrix $B$ with entries in
$\fe$ so that $\det(M+tB)\ne 0$ for every non-zero $t\in\fe$.
\enum

\heading{January 1990}
\num
\item Let $W_1$ and $W_2$ be subspaces of the finite dimensional
vector space $V$.  Record and prove a formula which relates $\dim
W_1$, $\dim W_2$, $\dim (W_1+W_2)$, $\dim (W_1\cap W_2)$.
\item Let $M$ be a symmetric $n\times n$ matrix with real number
entries.  Prove that there is an $n\times n$ matrix $N$ with real
entries such that $N^3=M$.
\item TRUE OR FALSE. (If the statement is true, then prove it.  If
the statement is false, then give a counterexample.)  If two
nilpotent matrices have the same rank, the same minimal polynomial
and the same characteristic polynomial, then they are similar.
\enum

\heading{August 1990} 
\num
\item Suppose that $T:V\to W$ is a injective linear transformation
over a field $\fe$.  Prove that $T^*:W^*\to V^*$ is surjective.
(Recall that $V^*=L(V,\fe)$ is the vector space of linear
transformations from $V$ to $\fe$.)
\item If $M$ is the $n\times n$ matrix
$$
M=\[\begin{array}{ccccc} x& a & a& \cdots & a \\
			a & x & a& \cdots &a \\
			a & a & x& \cdots &a\\
			\vdots& \vdots&\vdots&\ddots&\vdots\\
			a & a& a& \cdots &x\end{array}\]
$$
then prove that $\det M=[x+(n-1)a](x-a)^{n-1}$. 
\item Suppose that $T$ is a linear operator on a finite dimensional
vector space $V$ over a field $\fe$.  Prove that $T$ has a cyclic
vector if and only if
$$
\{U\in L(V,V): TU=UT\}=\{f(T): f\in \fe[x]\}.
$$
\item Let $T:\R^4\to \R^4$ be given by
$$
T(x_1,x_2,x_3,x_4)=(x_1-x_4,x_1,-2x_2-x_3-4x_4,4x_2+x_3)
$$
\num
\item Compute the characteristic polynomial of $T$.
\item Compute the minimal polynomial of $T$.
\item The vector space $\R^4$ is the direct sum of two proper
$T$-invariant subspaces.  Exhibit a basis for one of these
subspaces.
\enum
\enum

\heading{January 1991} 

\num
\item
Let $V$, $W$, and $Z$ be finite dimensional vector spaces over the
field $\fe$ and let $f:V\to W$ and $g:W\to Z$ be linear
transformations.  Prove that
$$
\mathop{\rm nullity}(g\circ f)\le \mathop{\rm nullity}( f)
+\mathop{\rm nullity}( g)
$$
\item Prove that 
$$
\det\[\begin{array}{ccc}A&0&0\\B&C&D\\ 0&0&E\end{array}\]
	=\det A\det C\det E
$$
where $A$, $B$, $C$, $D$ and $E$ are all square matrices.
\item Let $A$ and $B$ be $n\times n$ matrices with entries on the
field $\fe$ such that $A^{n-1}\ne0$, $B^{n-1}\ne 0$, and $A^n=B^n=0$.
Prove that $A$ and $B$ are similar, or show, with a counterexample,
that $A$ and $B$ are not necessarily similar.
\enum

\heading{August 1991}

\num
\item Let $A$ and $B$ be $n\times n$ matrices with entries from $\R$.
Suppose that $A$ and $B$ are similar over $\cn$.  Prove that they are
similar over $\R$.
\item Let $A$ be an $n\times n$ with entries from the field $\fe$.
Suppose that $A^2=A$.  Prove that the rank of $A$ is equal to the
trace of $A$.
\item TRUE OR FALSE. (If the statement is true, then prove it.  If the
statement is false, then give a counterexample.) Let $W$ be a vector
space over a field $\fe$ and let $\theta:V\to V'$ be a fixed
surjective transformation. If $g:W\to V'$ is a linear transformation
then there is linear transformation $h:W\to V$ such that $\theta\circ
h=g$.  \enum

\heading{January 1992}

\num\item
Let $V$ be a finite dimensional vector space and $A\in L(V,V)$.
\num \item Prove that there exists and integer $k$ such that
$\mathop{\rm ker}A^k=\mathop{\rm ker}A^{k+1}=\cdots$
\item Prove that there exists an integer $k$ such that $V=\mathop{\rm
ker}A^k\oplus \mathop{\rm image} A^k$.
\enum
\item Let $V$ be the vector space of $n\times n$ matrices over a field
$\fe$, and let $T:V\to V^*$ be defined by $T(A)(B)=\tr(A^tB)$.  Prove
that $T$ is an isomorphism.
\item Let $A$ be an $n\times n$ matrix and $A^k=0$ for some $k$.  Show
that $\det(A+I)=1$.
\item Let $V$ be a finite dimensional vector sauce over a field $\bf F$,
and $T$ a linear operator on $V$.  Suppose the minimal and
characteristic polynomials of of $T$ are the same power of an
irreducible polynomial $p(x)$.  Show that no non-trivial
$T$-invariant subspace of $V$ has a $T$-invariant complement. 
\enum
\heading{August 1992}
\num
\item Let $V$ be the vector space of all $n\times n$ matrices over a
field $\fe$, and let $B$ be a fixed $n\times n$ matrix that is not of
the form $cI$.  Define a linear operator $T$ on $V$ by $T(A)=AB-BA$.
Exhibit a not-zero element in the kernel of the transpose of $T$.
\item Let $V$ be a finite dimensional vector space over a field $\fe$
and suppose that $S$ and $T$ are triangulable operators on $V$.  If
$ST=TS$ prove that $S$ and $T$ have an eigenvector in common.
\item Let $A$ be an $n\times n$ matrix over $\cn$.  If trace $A^i=0$
for all $i>0$, prove that $A$ is nilpotent.
\enum
\heading{January 1993}

\num
\item Let $V$ be a finite dimensional vector space over a field $\fe$,
and let $T$ be a linear operator on $V$ so that $\mathop{\rm
rank}(T)=\mathop{\rm rank}(T^2)$.  Prove that $V$ is the direct sum of
the range of $T$ and the null space of $T$.

\item Let $V$ be the vector space of all $n\times n$ matrices over a
field $\fe$, and suppose that $A$ is in $V$.  Define $T:V\to V$ by
$T(AB)=AB$.  Prove that $A$ and $T$ have the same characteristic
values.
\item Let $A$ and $B$ be $n\times n$ matrices over the complex numbers.
\num
\item Show that $AB$ and $BA$ have the same characteristic values.
\item Are $AB$ and $BA$ similar matrices?
\enum
\item Let $V$ be a finite dimensional vector space over a field of
characteristic~0, and $T$ be a linear operator on $V$ so that
$\tr(T^k)=0$ for all $k\ge1$, where $\tr(\cdot)$ denotes the trace
function.  Prove that $T$ is a nilpotent linear map.
\item Let $A=[a_{ij}]$ be an $n\times n$ matrix over the field of
complex numbers such that
$$
|a_{ii}|>\sum_{j\ne i}|a_{ij}|\qquad \mbox{for} \qquad i=1\cd n.
$$
Then show that $\det A\ne 0$. ($\det$ denotes the determinant.)
\item Let $A$ be an $n\times n$ matrix, and let $\mathop{\rm adj}(A)$
denote the adjoint of $A$.  Prove the rank of $\mathop{\rm adj}(A)$ is
either $0$, $1$, or $n$.
\enum


\heading{August 1993}

\num
\item Let 
$$
A=\[\begin{array}{ccc}1&3&3\\3&1&3\\-3&-3&-5\end{array}\]
$$
\num
\item Determine the rational canonical form of $A$.
\item Determine the Jordan canonical form of $A$.
\enum
\item If 
$$
A=\[\begin{array}{ccc}0&1&0\\0&0&1\\0&0&0\end{array}\],
$$
then prove that there does not exist a matrix with $N^2=A$.
\item Let $A$ be a real $n\times n$ matrix which is symmetric, i.e.
$A^{ t}=A$.  Prove that $A$ is diagonalizable.
\item Give an example of two nilpotent matrices $A$ and $B$ such that 
\num
\item $A$ is not similar to $B$,
\item $A$ and $B$ have the same characteristic polynomial,
\item $A$ and $B$ have the same minimal polynomial, and
\item $A$ and $B$ have the same rank.
\enum
\enum

\heading{January 1994}
\num
\item Let $A$ be an $n\times n$ matrix over a field $\fe$.  Show that
$\fe^n$ is the direct sum of the null space and the range of $A$ if
and only if $A$ and $A^2$ have the same rank.
\item Let $A$ and $B$ be $n\times n$ matrices over a field $\fe$.
\num
\item Show $AB$ and $BA$ have the same eigenvalues.
\item Is $AB$ similar to $BA$?  (Justify your answer).
\enum
\item Given an exact sequence of finite-dimensional vector spaces 
$$
0 \stackrel{T_0}{\longrightarrow} V_1
 \stackrel{T_1}{\longrightarrow} V_2
\stackrel{T_2}{\longrightarrow}
 \cdots 
\stackrel{T_{n-2}}{\longrightarrow} V_{n-1}
\stackrel{T_{n-1}}{\longrightarrow} V_{n}
\stackrel{T_{n}}{\longrightarrow}0
$$
that is the range of $T_i$ is equal to the null space of $T_{i+1}$,
for all $i$.  What is the value of $\displaystyle
\sum_{i+1}^n(-1)^i\dim(V_i)$? (Justify your answer).
\item Let $\fe$ be a field with $q$ elements and $V$ be a
$n$-dimensional vector space over $\fe$.
\num
\item Find the number of elements in $V$.
\item Find the number of bases of $V$.
\item Find the number of invertible linear operators on $V$.
\enum
\item Let $A$ and $B$ be $n\times n$ matrices over a field $\fe$.
Suppose that $A$ and $B$ have the same trace and the same minimal
polynomial of degree $n-1$.  Is $A$ similar to $B$?  (Justify your
answer.)
\item Let $A=[a_{ij}]$ be an $n\times n$ matrix with $a_{ij}=1$ for
all $i$ and $j$.  Find its characteristic and minimal polynomial.

\enum

\heading{August 1994}

\num
\item Give an example of a matrix with real entries whose
characteristic polynomial is $x^5-x^4+x^2-3x+1$.
\item TRUE or FALSE. (If true prove it.  If false give a
counterexample.) Let $A$ and $B$ be $n\times n$ matrices with minimal
polynomial $x^4$.  If $\mathop{\rm rank} A=\mathop{\rm rank}B$, and 
$\mathop{\rm rank}A^2=\mathop{\rm rank}B^2$, then $A$ and $B$ are
similar.
\item Suppose that $T$ is a linear operator on a finite-dimensional
vector space $V$ over a field $\fe$.  Prove that the characteristic
polynomial of $T$ is equal to the minimal polynomial of $T$ if and
only if
$$
\{U\in L(V,V):TU=UT\}=\{f(T):f\in \fe[x]\}.
$$
\enum

\heading{January 1995}
\num\item \num
	\item Prove that if $A$ and $B$ are $3\times 3$ matrices over
a field $\fe$, a necessary and sufficient condition that $A$ and $B$ be
similar over $\fe$ is that that have the same characteristic and the
same minimal polynomial.
\item Give an example to show this is not true for $4\times 4$ matrices.
\enum
\item Let $V$ be the vector space of $n\times n$ matrices over a
field.  Assume that $f$ is a linear functional on  $V$ so that
$f(AB)=f(BA)$ for all $A,B\in V$, and $f(I)=n$.  Prove that $f$ is the
trace functional.
\item Suppose that $N$ is a $4\times 4$ nilpotent matrix over $\fe$
with minimal polynomial $x^2$.  What are  the possible rational
canonical forms for $n$?
\item Let $A$ and $B$ be $n\times n$ matrices over a field $\fe$.
Prove that $AB$ and $BA$ have the same characteristic polynomial.
\item Suppose that $\bf V$ is an $n$-dimensional vector space over $\fe$,
and $T$ is a linear operator on $\bf V$ which has $n$ distinct
characteristic values.  Prove that if $S$ is a linear operator on $\bf
V$ that commutes with $T$, then $S$ is a polynomial in $T$.
\enum

\heading{August 1995}
\num
\item Let $A$ and $B$ be $n\times n$ matrices over a field $\fe$.
Show that $AB$ and $BA$ have the same characteristic values in $\fe$.
\item Let $P$ and $Q$ be real  $n\times n$ matrices so that $P+Q=I$
and $\mathop{\rm rank}(P)+\mathop{\rm rank}(Q)=n$.  Prove that $P$ and
$Q$ are projections. ({\sc Hint:} Show that if $Px=Qy$ for some
vectors $x$ and $y$,  then $Px=Qy=0$.)
\item Suppose that $A$ is an  $n\times n$ real, invertible matrix.
Show that $A^{-1}$ can be expressed as a polynomial in $A$ with real
coefficients and with degree at most $n-1$.
\item Let
$$
A=\[\begin{array}{cccc}1&1&0&0\\1&1&1&0\\1&1&1&1\\1&1&1&1\end{array}\].
$$
Determine the rational canonical form and the Jordan canonical form
of $A$.
\item \num\item Give an example of two $4\times 4$ nilpotent matrices
which have the same minimal polynomial but are not similar.
\item Explain why $4$ is the smallest value that can be chosen for the
example in part (a), i.e. if $n\le 3$, any two nilpotent matrices
with the same minimal polynomial are similar.

\enum
\enum

\heading{January 1996}

\num
\item Let ${\mathcal P}_3$ be the vector space of all with coefficients
from $\R$ and of degree at most~$3$.  Define a linear
$T:\mathcal{P}_3\to \mathcal{P}_3$ by $(Tf)(x)=f(2x-6)$.  Is $T$
diagonalizable? Explain why.

\item Let $R$ be the ring of $n\times n$ matrices over the real
numbers.  Show that $R$ does not have any two sided ideals other than
$R$ and $\{0\}$.

\item Let $V$ be  a finite dimensional vector space and $A: V\to V$ a
linear map.  Suppose that $V=U\oplus W$ is  a direct sum decomposition
of $V$ into subspaces invariant under $A$.  Let $V^*$ be the dual
space of $V$ and let $A^t: V^*\to V^*$ be the transpose of $A$.
\num
\item Show that $V^*$ has a direct sum decomposition $V^*=X\oplus Y$
so that $\dim X=\dim U$ and $\dim Y =\dim W$ and both $X$ and $Y$ are
invariant under $A^t$.
\item Using part (a), or otherwise, prove that $A$ and $A^t$ are similar.
\enum
\enum

\heading{August 1996}
\num
\item Consider a linear operator on the space of $3\times 3$ matrices
defined by $S(A)=A-A^t$ where $A^t$ is the transpose of $A$.  Compute
the rank of $A$.
\item Let $V$ and $W$ be finite dimensional vector spaces over a field
$\mathbb F$, let $V^*$ and $W^*$ be the dual spaces to $V$ and $W$ and
let $T: V\to W$ be a linear map.
\num 
\item Give the definition of $V^*$ and show $\dim V=\dim V^*$.
\item If $S\subset V$ define the annihilator $S^\circ$ of $S$ in $V^*$
and prove it is a subspace of $V^*$.
\item Define the adjoint map $T^*: W^* \to V^*$.
\item Show that $\ker(T)^\circ = \image T^*$
\enum


\item Suppose that $A$ is a $3\times 3$ real orthogonal matrix, i.e.,
$A^t=A^{-1}$, with determinant~$-1$.  Prove that $-1$ is an eigenvalue
of for $A$.
\enum

\heading{January 1997}
\num
\item Let $M_{n\times n}$ be the vector space of all $n\times n$ real
matrices.
\num 
\item Show that every $A\in M_{n\times n}$ is similar to its transpose.
\item Is there a single invertible $S\in M_{n\times n}$ so that
$SAS^{-1}=A^t$ for all $A\in M_{n\times n}$?
\enum
\item Let $A$ be a $3\times 3$ matrix over the real numbers and assume
that $f(A)=0$ where $f(x)=x^2(x-1)^2(x-2)$.  Then give a complete
list of the possible values of $\det(A)$.
\item Show that for every polynomial $p(x)\in \mathbb{C}[x]$ of degree
$n$ there is a polynomial $q(x)$ of degree $\le n$ so that
$$
(x+1)^nf\Big(\frac{x-1}{x+1}\Big)=p(x).
$$
{\sc Hint:} Let $\mathcal{P}_n$ be the vector space of polynomials of
degree $\le n$ and for each $f(x)\in \mathcal{P}_n$ define
$(Sf)(x):=(x+1)^nf\big((x-1)/(x+1)\big)$.  Show that $S$ maps
$\mathcal{P}_n\to \mathcal{P}_n$ and is linear.  What is the null
space of $S$?
\enum
\pagebreak


\heading{August 1997}

\num
\item Let $V$ be a finite dimensional vector space and $L\in
\operatorname{Hom}(V,V)$ such that $L$ and $L^2$ have the same
nullity.  Show that $V=\operatorname{ker} L\oplus \operatorname{Im}L$.
\item Let $A$ be an $n\times n$ matrix and $n>1$.  Show that
$\text{adj}(\text{adj}(A))=\det(A^{n-2})A$.
\item Let $A=\displaystyle\[\begin{matrix} 1&1&0\\ -1&2&1\\
3&-6&6\end{matrix}\]$. Cmpute the rational cononical form and the
Jordon canonical form of $A$.
\item Let $A$ be an $n\times n$ real matrix such that $A^3=A$.  Show
that the rank of $A$ is greater than or equal to the trace of $A$.
\item Let $A=[a_{ij}]$ be a real $n\times n$ matrix with positive
diagonal entries such that 
$$
a_{ii}a_{jj}>\sum_{k\ne i}|a_{ik}|\sum_{l\ne j}|a_{il}|
$$ 
for all $i,j$.  Show that
$\det(A)>0$. {\sc Hint:} Show first that $\det(A)\ne 0$.
\enum
\heading{January 1998}
\num
\item For any nonzero scalar $a$, show that there are no real $n\times
n$ matrices $A$ and $B$ such that $AB-BA=aI$.
\item Let $V$ be a vector space over the rational numbers $\mathbb Q$
with $\dim V=6$ and let $T$ be a nonzero linear operator on $V$.
\num
\item If $f(T)=0$ for $f(x)=x^6+36 x^4-6x^2+12$, determine the
rational canonical form for $T$ (and prove your result is correct).
\item Is $T$ an automorphism of $V$?  If so describe $T^{-1}$; if not
describe why not.
\enum
\item Suppose that $A$ and $B$ are diagonalizable matrices over a
field $\mathbb F$. Prove that they are simultaneously diagonalizable,
that is there there exists an invertible matrix $P$ such that
$PAP^{-1}$ and $PBP^{-1}$ are both diagonal, if and only if $AB=BA$.
\enum
%\pagebreak



\heading{August 1998}

\num
\item $V$ be a finite dimensional vector space and let $W$ be a
subspace of $V$.  Let $\mathcal{L}(V)$ the set of linear operators on
$V$ and set $Z=\{T\in \mathcal{L}(V): W\subseteq
\operatorname{ker}(T)\}$. Prove that $Z$ is a subspace of
$\mathcal{L}(V)$ and compute its dimension in terms of the the
dimensions of $V$ and $W$.

\item Let $V$ be a finite dimensional vector space and
$\mathcal{L}(V)$ the set of linear operators on $V$.  Suppose $T\in
\mathcal{L}(V)$.  Suppose that
$$
V=V_1\oplus V_2\oplus \cdots \oplus V_r
$$
where $V_i$ is $T$ invariant for each $i\in \{1,\dots,k\}$.  Let
$m(x)$ be the minimal polynomial of $T$ and $m_i(x)$ the minimal
polynomial of $T$ restricted to $V_i$, for each $i\in \{1,\dots,k\}$.
How is $m(x)$ related to the set $\{m_1(x),\dots,m_r(x)\}$.

\item Let $V$ be a finite dimensional vector space and
$\mathcal{L}(V)$ the set of linear operators on $V$.  Let $S,T\in
\mathcal{L}(V)$ so that $S+T=I$ and $\dim \operatorname{range} S +\dim
\operatorname{range} T=\dim V$.  Prove that $V=\operatorname{range}
S\oplus \operatorname{range} T$ and that $ST=TS=0$.

\item Let $A$ and $B$ be be $n\times n$ matrices.  Suppose that $A^k$
and $B^k$ have the same minimal polynomials and the same
characteristic polynomials for $k=1,2,$ and $3$.  Must $A$ and $B$ be
similar?  If so prove it.  If not, give a counterexample.
\enum 

\heading{January 1999}

\num
\item Let $V$ be a finite dimensional vector space and let $T: V\to V$
be a linear transformation which is not zero and is not an
isomorphism.  Prove there is exists a linear transformation $S$ so that
$ST=0$, but $TS\ne 0$.

\item Let $T$ be a linear operator on the finite dimensional vector
space $V$.  Prove that if $T^2=T$, then $V=\operatorname{ker} T\oplus
\operatorname{image} T$.

\item Let $S$ and $T$ be $5\times 5$ nilpotent matrices with
$\operatorname{rank}S =\operatorname{rank}T$ and
$\operatorname{rank}S^2 =\operatorname{rank}T^2$.  Are $S$ and $T$
necessarily similar?  Prove or give a counterexample.

\item Let $A$ and $B$ be $n\times n$ matrices over $\mathbb C$ with
$AB=BA$.  Prove $A$ and $B$ have a common eigenvector.  Do $A$ and $B$
have a common eigenvalue.
\enum


\heading{August 1999}
\num
\item Make a list, sa long as possible, of square matrices over
$\mathbb C$ such that
\num 
\item Each matrix on the list has characteristic polynomial
$(x-2)^4(x-3)^4$,
\item Each matrix on the list has minimal polynomial $(x-2)^2(x-3)^2$,
and,
\item No matrix on the list is similar to a matrix occurring elsewhere
on the list.
\enum
Demonstrate that your list has all the desired attributes.

\item Let $A$ and $B$ be nilpotent matrices over $\mathbb C$.
\num 
\item
Prove that if $AB=BA$, then $A+B$ is nilpotent.
\item
Prove that $I-A$ is invertible.
\enum

\item
Let $V$ be a finite dimensional vector space. Recall that for
$X\subseteq V$ the set $X^\circ $ is defined to be $\{ f\ |\ \text{$f$
is a linear functional of $V$ and $f(x)=0$ for all $x\in X$}$.  Let
$U$ and $W$ be subspaces of $V$.  Prove the following
\num
\item $(U+W)^\circ =U^\circ\cap W^\circ$.
\item $U^\circ + W^\circ =(U\cap W)^\circ$.
\enum

\enum





\heading{Some Other Problems}

\num
\item {\bf This is a very basic and important fact.}
Let $V$ be a finite dimensional vector space and $f$ and $g$ two
linear functionals on $V$.  If $\ker f=\ker g$ show  $g$ is a scalar
multiple of $f$.
\item This problem makes explicit some facts that are used several times in
solving some of the problems above.
\num \item  Prove that if $V$ is a finite dimensional vector space
over the field $\fe$ and $T\in L(V,V)$ and $V$ is cyclic for $T$ that any
$S\in L(V,V)$ that commutes with $T$ is a polynomial in $T$.  That is $ST=
TS$ implies that $S=p(T)$ for some $p(x)\in\fe[x]$.  {\sc Hint:} Let $\dim
V=n$.  Then because $V$ is cyclic for $T$ there is a vector $v_0\in V$ so
that $v_0, Tv_0\cd T^{n-1}v_0$ is a basis for $V$.  Thus there are
scalars $a_0,a_1\cd a_{n-1}$ so that $Sv_0=a_0v_0+a_1Tv_0+a_2T^2v_0+\cdots
+a_{n-1}T^{n-1}v_0$.  Then letting $p(x)=a_0+a_1x+a_2x^2+\cdots
a_{n-1}x^{n-1}$ we have $Sv_0=p(T)v_0$.  Now use that $S$ commutes with $T$
(and thus also $p(T)$) to show that $S T^iv_0=p(T)T^iv_0$ for $i=0,1\cd
n-1$.  Thus the two linear maps $S$ and $p(T)$ agree on a basis, whence
are equal.
\item If the minimal polynomial $f(x)$ of $T$ has $\mathop{\rm deg}
f(x)=\dim V$ then $V$ is cyclic for $T$.  {\sc Hint:} I don't know any
particularly easy way to do this.  The basic idea is to factor
$f(x)=p_1(x)^{k_1}\cdots p_l(x)^{k_l}$ into powers of primes and consider 
the corresponding primary decomposition $V=\ker(p_1(T)^{k_1})\oplus\cdots\oplus
\ker(p_l(T)^{k_l})$ and show that if $\mathop{\rm deg} f(x)=\dim V$ then
each of the primary factors $\ker(p_i(T)^{k_i})$ is cyclic (this in turn uses
that each of the $\ker(p_i(T)^{k_i})$ is a sum of cyclic subspaces).  Now
let $v_i$ be a cyclic for $T$ in $\ker(p_i(T)^{k_i})$ for $i=1\cd l$.  Then
show the vector $v_0=v_1+v_2+\cdots +v_l$ is cyclic for $T$. An alternative is 
to use Smith normal form.
\enum
\item Let $\lambda_1,\lambda_2\cd \lambda_n$ be distinct elements of the
field $\fe$.  Then the matrix
$$
A=\[\begin{array}{cccc}1&1&\cdots&1\\ 
\lambda_1& \lambda_2&\cdots &\lambda_n\\
\lambda_1^2&\lambda_2^2&\cdots &\lambda_n^2\\
\vdots&\vdots&\ddots&\vdots\\
\lambda_1^{n-1}&\lambda_2^{n-1}&\cdots&\lambda_n^{n-1}\end{array}\]
$$
is invertible.  {\sc Hint:}  If $A$ is singular then it has rank less than
$n$ and thus there is a nontrivial linear relation between the rows of $A$.
This would in turn imply that there is a nonzero polynomial $p(x)$ of
degree $\le n-1$ that had the $n$ scalars $\lambda_1,\lambda_2\cd
\lambda_n$ as roots. But this is impossible.
\item {\bf This is another set of facts that anyone who has had a graduate
linear algebra class should know.} Let $D$ be a diagonal matrix that has
all its diagonal elements distinct.  That is,
$$
D=\mathop{\rm diag}(\lambda_1,\lambda_2\cd \lambda_n):=
\[\begin{array}{cccc}\lambda_1&0&\cdots&0\\
			0&\lambda_2&\cdots&0\\
			\vdots&\vdots&\ddots&\vdots\\
			0&0&\cdots&\lambda_n\end{array}\]
\qquad\mbox{where $\lambda_i\ne\lambda_j$ for $i\ne j$.}
$$
Then show 
\num
\item The only matrices that commute with $D$ are diagonal matrices.
\item If $C$ is any other diagonal matrix, 
then $C$ is a polynomial in $D$.
\item If $A$ is any matrix that commutes with $D$, then $A$ is a polynomial
in $D$.
\item There is a cyclic vector for $T$.  {\sc Hint:}  Let $e_1\cd e_n$ be
the standard coordinate vectors.  Then as $D$ is diagonal
$De_i=\lambda_ie_i$.  Let $v=a_1e_1+a_2e_2+\cdots+a_ne_n$.  Then show that
$v$ is a cyclic vector for $D$ if and only if $a_i\ne0$ for all $i$\/ (one
way to do this is use the last problem).  In particular
$v=e_1+e_2+\cdots+e_n$ is a cyclic vector for $T$. Or use that the 
minimal and characteristic polynomials for $D$ are the same, together with 
problem 2 above.
\enum
\item This is another standard problem.  Let $V$ be a finite dimensional
vector space over a field $\fe$ and let $T\in L(V,V)$.  Let
$\lambda$ be an eigenvalue of $T$ and let $V_\lambda:=\{v\in V:Tv=\lambda
v\}$ be the corresponding eigenspace.
\num \item Let $S\in L(V,V)$ commute with $T$.  Then show that $V_\lambda$
is invariant under $S$.  (That is show $v\in V_\lambda$ implies $Sv\in
V_\lambda$.)  
\item Show that if $A$ and $B$ are  $n\times n$ matrices over the complex
numbers that commute they have a common eigenvector.  {\sc Hint:} As $A$ is a
complex matrix it has at least one eigenvalue $\lambda$.  Let $V_\lambda$
be the corresponding eigenspace.  Then by what we have just done
$V_\lambda$ is invariant under $B$.  But then the restriction of $B$ to
$V_\lambda$ has an eigenvector in $V_\lambda$.  
\item This is a different way of looking at Problem~4 above.  Assume 
$V$ has an basis of eigenvectors $e_1,e_2\cd e_n$ of eigenvectors of $T$,
that is $Te_i=\lambda e_i$.  Also assume the eigenvalues are distinct:
$\lambda_i\ne\lambda_j$ for $i\ne j$.  Then
show if $S$ commutes with $T$ then for some scalars $c_i$ there holds
$Se_i=c_ie_i$, and thus $S$ is also diagonal in the basis $e_1\cd e_n$.
{\sc Hint:}  Let $V_{\lambda_i}:=\{v:Te_i=\lambda_iv\}$.  Then by the
assumptions $V_{\lambda_i}$ is one dimensional with basis $e_i$.  
Part~(a) of this problem then implies that $V_{\lambda_i}$ is invariant
under $S$.  As $V_{\lambda_i}$ is one dimensional this in turn implies
$e_i$ is an eigenvector of $S$.
\enum
\enum





\end{document}







